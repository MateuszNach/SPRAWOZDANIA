\documentclass[10pt,a4paper]{article}
\usepackage[utf8]{inputenc}
\usepackage{amsmath}
\usepackage{amsfonts}
\usepackage{graphicx}
\usepackage{amssymb}
\usepackage{polski}
\usepackage{color}
\author{Mateusz Nachyła}
\title{Laboratorium IOT}
\begin{document}
\maketitle
\begin{center}
\begin{tabular}{|c|c|} \hline
Politechnika Świętokrzyska & Kielce \\
Temat & Latex \\
Laboratorium & 1 \\
\hline \hline
Przedmiot & IOT \\ \hline
\end{tabular}
\end{center}

\begin{center}

{\Huge Przykładowe zdania z różną czcionką oraz kolorem \\}

\begin{tiny}
Laboratorim 1 Latex Nachyła Mateusz\\
\end{tiny}

\begin{footnotesize}
Laboratorim 1 Latex Nachyła Mateusz\\
\end{footnotesize}

\begin{Large}
Laboratorim 1 Latex Nachyła Mateusz\\
\end{Large}

\begin{Large}
\textcolor{red}{LAboratorium IOT}\\
\end{Large}

\begin{huge}
\textcolor{blue}{LAboratorium IOT}\\
\end{huge}

\pagecolor{yellow}
\section{BIBLIOGRAFIA}
\end{center}





\cite{wiki:latex}
\bibliographystyle{ieeetr}
\bibliography{Bibliografia}

\newpage
\begin{center}
\begin{Huge}
Wnioski\\
\end{Huge}
\end{center}
\begin{Large}
LaTeX jest to oprogramowanie do zautomatyzowanego składu tekstu, a także związany z nim język znaczników, służący do formatowania dokumentów tekstowych i tekstowo-graficznych.
W sposób automatyczny tworzone są: spisy treści, ilustracji oraz tabel, numerowanie i referencje do rozdziałów i podrozdziałów, numerowanie i referencje elementów takich jak wzory i rysunki,skorowidze,bibliografia.
Użyte przeze mnie środowisko którym się posługiwałem to Miktex ver 2.9 oraz TexMaker ver 4.5 .
\end{Large}


\end{document}
\end{document}




